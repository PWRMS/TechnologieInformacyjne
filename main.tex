\documentclass{article}
\usepackage[utf8]{inputenc}
\usepackage[polish]{babel}
\usepackage{graphicx}
\usepackage[T1]{fontenc}
\usepackage{amsmath}
\usepackage[hmargin = 1.25 in, vmargin = 1 in]{geometry}
\usepackage{booktabs}
\usepackage{longtable}

\usepackage{pgfplots}
\usepackage{siunitx}
\usepackage{tikz}


\pgfplotsset{compat=newest}
\usepgfplotslibrary{units} 


\title{learn tex}
\author{Michał Szmidt}
\date{December 2022}

\begin{document}
	
	\maketitle
	
	\section{Wstęp teoretyczny}
	Spadek swobodny – ruch odbywający się wyłącznie pod wpływem ciężaru (siły grawitacji), bez oporów ośrodka. Jeżeli spadek ma miejsce z małej wysokości w pobliżu powierzchni Ziemi i dotyczy ciała o stosunkowo dużej gęstości i aerodynamicznym kształcie (np. kuli), wówczas ruch takiego ciała można z dobrym przybliżeniem traktować jak ruch jednostajnie przyspieszony z przyspieszeniem ziemskim g bez prędkości początkowej. Ruch ten opisuje kinematyczne równanie ruchu w poniższej postaci.
	
	\begin{equation} \label{eq1}
		\begin{split}
			h\left(t\right)=h_0-\frac{gt^2}{2}
		\end{split}
	\end{equation}
	
	\section{Opis Eksperymentu}
	
	Figure~\ref{schemat eksperymentu}
	\begin{figure}[!htbp]
		\centering
		\includegraphics{opis_eksperymentu.png}
		\caption{schemat eksperymentu}
		\label{schemat eksperymentu}
	\end{figure}
	
	\section{Przebieg doświadczenia}
	
	\begin{enumerate}
		\item Nakrętki wiążemy nicią jak na rysunku. Wartość l może wynosić 30 cm, wtedy 4 l = 120 cm, 9 l = 270 cm itd.
		\item Trzymając nić nad najwyższą nakrętką, rozciągamy ją tak, aby najniższa nakrętka dotykała podłoża.
		\item Puszczamy nić, rejestrujemy uderzenia nakrętek o podłoże.
	\end{enumerate}
	
	\section{Wyniki pomiarów}
	\begin{center}
		\begin{longtable}{|l|l|l|l|}
			
			
			\hline \multicolumn{1}{|c|}{\textbf{t(s)}} & \multicolumn{1}{c|}{\textbf{s(m)}} &\multicolumn{1}{c|}{\textbf{$\Delta$ s}} & \multicolumn{1}{c|}{\textbf{\^{s}}} \\ \hline 
			\endfirsthead
			
			\multicolumn{4}{c}%
			{{\bfseries \tablename\ \thetable{} -- Ciąg dalszy}} \\
			\hline \multicolumn{1}{|c|}{\textbf{t(s)}} &
			\multicolumn{1}{c|}{\textbf{s(m)}} &
			\multicolumn{1}{c|}{\textbf{$\Delta$ s}} &
			\multicolumn{1}{c|}{\textbf{\^{s}}} \\ \hline 
			\endhead
			
			\hline \multicolumn{4}{|r|}{{Kontynuacja na nst. stronie}} \\ \hline
			\endfoot
			
			\hline \hline
			\endlastfoot
			\centering
			0.1 & 0.05 & 4.92 & -4.87 \\ 
			0.2 & 0.2 & 1.19 & -1 \\ 
			0.3 & 0.44 & -0.26 & 0.7 \\ 
			0.4 & 0.78 & 7.39 & -6.61 \\ 
			0.5 & 1.23 & -2.1 & 3.32 \\ 
			0.6 & 1.76 & -2.13 & 3.9 \\ 
			0.7 & 2.4 & -1.3 & 3.7 \\ 
			0.8 & 3.14 & 3.88 & -0.74 \\ 
			0.9 & 3.97 & -4.35 & 8.32 \\ 
			1 & 4.9 & -0.32 & 5.22 \\ 
			1.1 & 5.93 & 1.4 & 4.53 \\ 
			1.2 & 7.06 & 1.28 & 5.77 \\ 
			1.3 & 8.28 & -0.09 & 8.38 \\ 
			1.4 & 9.6 & -2.16 & 11.76 \\ 
			1.5 & 11.03 & 0.4 & 10.62 \\ 
			1.6 & 12.54 & 2.15 & 10.39 \\ 
			1.7 & 14.16 & -1.85 & 16.01 \\ 
			1.8 & 15.88 & 2.99 & 12.89 \\ 
			1.9 & 17.69 & 0.49 & 17.2 \\ 
			2 & 19.6 & -7.03 & 26.63 \\ 
			2.1 & 21.61 & 3.64 & 17.97 \\ 
			2.2 & 23.72 & -2.56 & 26.28 \\ 
			2.3 & 25.92 & 3.34 & 22.58 \\ 
			2.4 & 28.22 & -1.21 & 29.44 \\ 
			2.5 & 30.63 & -3.71 & 34.33 \\ 
			2.6 & 33.12 & -1.46 & 34.59 \\ 
			2.7 & 35.72 & 4.84 & 30.88 \\ 
			2.8 & 38.42 & 2.26 & 36.16 \\ 
			2.9 & 41.21 & -1.48 & 42.69 \\ 
			3 & 44.1 & 1.22 & 42.88 \\ 
			3.1 & 47.09 & 2.18 & 44.91 \\ 
			3.2 & 50.18 & 3.2 & 46.98 \\ 
			3.3 & 53.36 & -2.66 & 56.02 \\ 
			3.4 & 56.64 & 1.98 & 54.66 \\ 
			3.5 & 60.03 & 4.51 & 55.51 \\ 
			3.6 & 63.5 & 3.42 & 60.08 \\ 
			3.7 & 67.08 & 7.2 & 59.88 \\ 
			3.8 & 70.76 & 1.26 & 69.49 \\ 
			3.9 & 74.53 & -1.15 & 75.68 \\ 
			4 & 78.4 & -4.94 & 83.34 \\ 
			4.1 & 82.37 & 0.18 & 82.19 \\ 
			4.2 & 86.44 & -3.23 & 89.66 \\ 
			4.3 & 90.6 & 1.74 & 88.86 \\ 
			4.4 & 94.86 & -0.77 & 95.64 \\ 
			4.5 & 99.23 & -2.63 & 101.85 \\ 
			4.6 & 103.68 & 0.39 & 103.3 \\ 
			4.7 & 108.24 & -2.97 & 111.21 \\ 
			4.8 & 112.9 & 0.2 & 112.7 \\ 
			4.9 & 117.65 & 1.71 & 115.94 \\ 
			5 & 122.5 & 9.22 & 113.28 \\ 
			5.1 & 127.45 & -3.3 & 130.75 \\ 
			5.2 & 132.5 & 2.84 & 129.65 \\ 
			5.3 & 137.64 & 3.45 & 134.19 \\ 
			5.4 & 142.88 & -0.88 & 143.77 \\ 
			5.5 & 148.23 & -8.18 & 156.4 \\ 
			5.6 & 153.66 & -3.45 & 157.11 \\ 
			5.7 & 159.2 & 2.02 & 157.19 \\ 
			5.8 & 164.84 & 4.55 & 160.29 \\ 
			5.9 & 170.57 & 3.05 & 167.52 \\ 
			6 & 176.4 & -1.64 & 178.04 \\ 
			6.1 & 182.33 & -0.7 & 183.03 \\ 
			6.2 & 188.36 & 6.75 & 181.61 \\ 
			6.3 & 194.48 & -3.55 & 198.03 \\ 
			6.4 & 200.7 & -3.31 & 204.01 \\ 
			6.5 & 207.03 & -0.23 & 207.26 \\ 
			6.6 & 213.44 & -3.1 & 216.54 \\ 
			6.7 & 219.96 & -2.34 & 222.3 \\ 
			6.8 & 226.58 & 5.36 & 221.22 \\ 
			6.9 & 233.29 & -0.25 & 233.53 \\ 
			7 & 240.1 & 0.09 & 240.01 \\ 
			7.1 & 247.01 & -1.7 & 248.71 \\ 
			7.2 & 254.02 & -0.49 & 254.51 \\ 
			7.3 & 261.12 & -0.68 & 261.8 \\ 
			7.4 & 268.32 & 2.15 & 266.17 \\ 
			7.5 & 275.63 & 1.04 & 274.58 \\ 
			7.6 & 283.02 & 0.78 & 282.25 \\ 
			7.7 & 290.52 & -4.12 & 294.64 \\ 
			7.8 & 298.12 & 0.89 & 297.23 \\ 
			7.9 & 305.81 & -0.78 & 306.59 \\ 
			8 & 313.6 & -1.78 & 315.38 \\ 
			8.1 & 321.49 & -3.46 & 324.95 \\ 
			8.2 & 329.48 & -5.93 & 335.41 \\ 
			8.3 & 337.56 & 1.17 & 336.39 \\ 
			8.4 & 345.74 & 2.64 & 343.11 \\ 
			8.5 & 354.03 & 2.29 & 351.74 \\ 
			8.6 & 362.4 & -8.26 & 370.66 \\ 
			8.7 & 370.88 & -0.13 & 371.01 \\ 
			8.8 & 379.46 & 4.68 & 374.77 \\ 
			8.9 & 388.13 & 1.31 & 386.82 \\ 
			9 & 396.9 & 2.06 & 394.84 \\ 
			9.1 & 405.77 & 0.26 & 405.51 \\ 
			9.2 & 414.74 & -1.36 & 416.09 \\ 
			9.3 & 423.8 & 0.6 & 423.2 \\ 
			9.4 & 432.96 & 2 & 430.97 \\ 
			9.5 & 442.23 & -4.7 & 446.92 \\ 
			9.6 & 451.58 & 7.36 & 444.23 \\ 
			9.7 & 461.04 & 3.23 & 457.81 \\ 
			9.8 & 470.6 & 1.39 & 469.21 \\ 
			9.9 & 480.25 & -0.73 & 480.97 \\ 
			
		\end{longtable}
	\end{center}
	
	\begin{filecontents*}{short.csv}
		ts,sm
		0.1,0.05
		0.2,0.2
		0.3,0.44
		0.4,0.78
		0.5,1.23
		0.6,1.76
		0.7,2.4
		0.8,3.14
		0.9,3.97
		1,4.9
		1.1,5.93
		1.2,7.06
		1.3,8.28
		1.4,9.6
		1.5,11.03
		1.6,12.54
		1.7,14.16
		1.8,15.88
		1.9,17.69
		2,19.6
		2.1,21.61
		2.2,23.72
		2.3,25.92
		2.4,28.22
		2.5,30.63
		2.6,33.12
		2.7,35.72
		2.8,38.42
		2.9,41.21
		3,44.1
		3.1,47.09
		3.2,50.18
		3.3,53.36
		3.4,56.64
		3.5,60.03
		3.6,63.5
		3.7,67.08
		3.8,70.76
		3.9,74.53
		4,78.4
		4.1,82.37
		4.2,86.44
		4.3,90.6
		4.4,94.86
		4.5,99.23
		4.6,103.68
		4.7,108.24
		4.8,112.9
		4.9,117.65
		5,122.5
		5.1,127.45
		5.2,132.5
		5.3,137.64
		5.4,142.88
		5.5,148.23
		5.6,153.66
		5.7,159.2
		5.8,164.84
		5.9,170.57
		6,176.4
		6.1,182.33
		6.2,188.36
		6.3,194.48
		6.4,200.7
		6.5,207.03
		6.6,213.44
		6.7,219.96
		6.8,226.58
		6.9,233.29
		7,240.1
		7.1,247.01
		7.2,254.02
		7.3,261.12
		7.4,268.32
		7.5,275.63
		7.6,283.02
		7.7,290.52
		7.8,298.12
		7.9,305.81
		8,313.6
		8.1,321.49
		8.2,329.48
		8.3,337.56
		8.4,345.74
		8.5,354.03
		8.6,362.4
		8.7,370.88
		8.8,379.46
		8.9,388.13
		9,396.9
		9.1,405.77
		9.2,414.74
		9.3,423.8
		9.4,432.96
		9.5,442.23
		9.6,451.58
		9.7,461.04
		9.8,470.6
		9.9,480.25
	\end{filecontents*}

	\begin{filecontents*}{error.csv}
ts,blad
0.1,-4.87
0.2,-1
0.3,0.7
0.4,-6.61
0.5,3.32
0.6,3.9
0.7,3.7
0.8,-0.74
0.9,8.32
1,5.22
1.1,4.53
1.2,5.77
1.3,8.38
1.4,11.76
1.5,10.62
1.6,10.39
1.7,16.01
1.8,12.89
1.9,17.2
2,26.63
2.1,17.97
2.2,26.28
2.3,22.58
2.4,29.44
2.5,34.33
2.6,34.59
2.7,30.88
2.8,36.16
2.9,42.69
3,42.88
3.1,44.91
3.2,46.98
3.3,56.02
3.4,54.66
3.5,55.51
3.6,60.08
3.7,59.88
3.8,69.49
3.9,75.68
4,83.34
4.1,82.19
4.2,89.66
4.3,88.86
4.4,95.64
4.5,101.85
4.6,103.3
4.7,111.21
4.8,112.7
4.9,115.94
5,113.28
5.1,130.75
5.2,129.65
5.3,134.19
5.4,143.77
5.5,156.4
5.6,157.11
5.7,157.19
5.8,160.29
5.9,167.52
6,178.04
6.1,183.03
6.2,181.61
6.3,198.03
6.4,204.01
6.5,207.26
6.6,216.54
6.7,222.3
6.8,221.22
6.9,233.53
7,240.01
7.1,248.71
7.2,254.51
7.3,261.8
7.4,266.17
7.5,274.58
7.6,282.25
7.7,294.64
7.8,297.23
7.9,306.59
8,315.38
8.1,324.95
8.2,335.41
8.3,336.39
8.4,343.11
8.5,351.74
8.6,370.66
8.7,371.01
8.8,374.77
8.9,386.82
9,394.84
9.1,405.51
9.2,416.09
9.3,423.2
9.4,430.97
9.5,446.92
9.6,444.23
9.7,457.81
9.8,469.21
9.9,480.97


 	\end{filecontents*}
	
	\begin{figure}[htbp!]
		\begin{center}
			\begin{tikzpicture}
				\begin{axis}[
                    no markers,
					width=\linewidth, % Scale the plot to \linewidth
					grid=major, 
					grid style={dashed,gray!30},
					xlabel=s, % Set the labels
					ylabel=t,
					x unit=\si{\meter}, % Set the respective units
					y unit=\si{\second},
					xmin=0, xmax=10,
					ymin=0, ymax=500,
					ytick={0,100,200,300,400,500},
					xtick={0,1,2,3,4,5,6,7,8,9,10},
					legend pos=north west
					]
					\addplot+[line width=2pt,mark size=6pt] table[
                            x=ts,
                            y=sm, 
                            col  sep=comma
                            ] {short.csv};

                    \addplot+[
                    only marks,
                    mark=square,
                    mark options={scale=1,fill=white, fill opacity=0}
                    ] table[
                            x=ts,
                            y=blad,
                            col  sep=comma
                            ] {error.csv};
                    \addlegendentry{\(4.9*x^2\)}
				\end{axis}
			\end{tikzpicture}
			\caption{Wykres 1.}
		\end{center}
	\end{figure}
	
	
	\begin{enumerate}
		\item Swobodnie spadające ciała poruszają się ruchem jednostajnie przyspieszonym, a w tym ruchu droga jest proporcjonalna do kwadratu czasu. 
		\item Nakrętka z wysokości h spadała przez czas t, z wysokości 4 h – przez czas 2 t, z wysokości 9 h – przez czas 3 t. 
		\item Potwierdza to, że droga jest proporcjonalna do kwadratu czasu. 
		Udowodniliśmy, że swobodnie spadające ciała poruszają się ruchem jednostajnie przyspieszonym    
	\end{enumerate}
	
\end{document}
