\documentclass{article}
\usepackage[utf8]{inputenc}
\usepackage[polish]{babel}
\usepackage{graphicx}
\usepackage[T1]{fontenc}
\usepackage{amsmath}
\usepackage[hmargin = 1.25 in, vmargin = 1 in]{geometry}
\usepackage{booktabs}

\title{learn tex 0}
\author{Michał Szmidt}
\date{December 2022}

\begin{document}

\maketitle

\section{Wstęp teoretyczny}
Spadek swobodny – ruch odbywający się wyłącznie pod wpływem ciężaru (siły grawitacji), bez oporów ośrodka. Jeżeli spadek ma miejsce z małej wysokości w pobliżu powierzchni Ziemi i dotyczy ciała o stosunkowo dużej gęstości i aerodynamicznym kształcie (np. kuli), wówczas ruch takiego ciała można z dobrym przybliżeniem traktować jak ruch jednostajnie przyspieszony z przyspieszeniem ziemskim g bez prędkości początkowej. Ruch ten opisuje kinematyczne równanie ruchu w poniższej postaci.

\begin{equation} \label{eq1}
\begin{split}
h\left(t\right)=h_0-\frac{gt^2}{2}
\end{split}
\end{equation}

\section{Opis Eksperymentu}

Figure~\ref{schemat eksperymentu}
\begin{figure}[!htbp]
\centering
\includegraphics{opis_eksperymentu.png}
\caption{schemat eksperymentu}
\label{schemat eksperymentu}
\end{figure}

\section{Przebieg doświadczenia}

\begin{enumerate}
\item Nakrętki wiążemy nicią jak na rysunku. Wartość l może wynosić 30 cm, wtedy 4 l = 120 cm, 9 l = 270 cm itd.
\item Trzymając nić nad najwyższą nakrętką, rozciągamy ją tak, aby najniższa nakrętka dotykała podłoża.
\item Puszczamy nić, rejestrujemy uderzenia nakrętek o podłoże.
\end{enumerate}

\section{Wyniki pomiarów}

\begin{table}[!htbp]
    \centering
    \begin{tabular}{|c|c|c|}
    \hline
        ~ & ~ & ~ \\ \hline
        ~ & q & ~ \\ 
        ~ & q & ~ \\ 
        ~ & ~ & ~ \\ 
        ~ & ~ & ~ \\ 
        ~ & ~ & ~ \\ 
        ~ & ~ & ~ \\ 
        ~ & ~ & ~ \\ 
        ~ & ~ & ~ \\ 
        ~ & ~ & ~ \\ 
        ~ & ~ & ~ \\ 
        ~ & ~ & ~ \\ 
        ~ & ~ & ~ \\ 
        ~ & ~ & ~ \\ 
        ~ & ~ & ~ \\ 
        ~ & ~ & ~ \\ 
        ~ & ~ & ~ \\ 
        ~ & ~ & ~ \\ 
        ~ & q & ~ \\ 
        ~ & ~ & ~ \\ \hline
    \end{tabular}
\end{table}

Tu będą wykresy

\section{Wnioski}

\begin{enumerate}
\item Swobodnie spadające ciała poruszają się ruchem jednostajnie przyspieszonym, a w tym ruchu droga jest proporcjonalna do kwadratu czasu. 
\item Nakrętka z wysokości h spadała przez czas t, z wysokości 4 h – przez czas 2 t, z wysokości 9 h – przez czas 3 t. 
\item Potwierdza to, że droga jest proporcjonalna do kwadratu czasu. 
Udowodniliśmy, że swobodnie spadające ciała poruszają się ruchem jednostajnie przyspieszonym    
\end{enumerate}

\end{document}